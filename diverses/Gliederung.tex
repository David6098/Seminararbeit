\documentclass[12pt]{article}
\usepackage{scrlayer-scrpage}
\usepackage{biblatex}
\usepackage[ngerman]{babel}

\begin{document}
\ihead{Seitz David}
\chead{W-Seminar 1Inf11}
\ohead{\today}
\section*{Voraussichtliche Gliederung der W-Seminararbeit: Entwicklung und Implementierung eines KI-basierten Spielers des Spiels Space-Invaders }
\subsection*{Einleitung}
In der Einleitung soll über ein verwandtes Thema, wie zum Beispiel Deep Blue, ein Schachcomputer, an meine KI herangeführt werden. Sie soll Parallelen zu diesen Programmen herstellen, sowie einen Ausblick geben was in dieser Seminararbeit auf einen zukommt.  
\subsection*{Hauptteil}
 Im Hauptteil der Seminararbeit soll Schritt für Schritt die Künstliche Intelligenz gebaut und erläutert werden, dass heißt der Aufbau der KI, die Tainingsmethoden der KI und die generelle Grundidee wird erklärt. Er soll in einzelne Kapitel untergliedert werden, in welchen die einzelne Aspekte wie zum Beispiel das Preprocessing der Inputdaten oder der Algorithmus des bestärkenden Lernen erläutert werden und wie diese auf das Projekt angewendet werden. Auch wird es ein kurzes Kapitel darüber geben wie  das Spiel -Space Invaders- eigentlich funktioniert. Ein weiteres Kapitel wird von dem Grundgerüst einer KI handeln, also dem zugrundeliegenden Neuronalen Netz und der Outputfunktion der einzelnen Neuronen, obwohl abzusehen bleibt ob dies auch auf zwei kürzere Kapitel aufgeteilt werden könnte. Ich bin mir sicher das im Laufe des Programmierens der Seminararbeit mehr Kapitel hinzukommen werden, welche dann von einem klar abgegrenzten Teilbereich handeln werden. 
\subsection*{Schluss}
Der Schluss soll eine Zusammenfassung der Seminararbeit enthalten und zusammenfassen in wie fern die in der Einleitung ausgelegten Ziele erreicht, überschritten oder nicht erfüllt werden werden konnten.
\end{document}
