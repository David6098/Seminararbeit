\chapter{Anleitung}

Diese Vorlage besteht aus folgenden Dateien, welche alle im gleichen Verzeichnis liegen:

\begin{center}
\begin{tabular}{ll}
Dateiname & Inhalt\\\hline
Seminararbeit.tex & Struktur des Gesamtdokumentes\\
Quellen.bin & Literaturverzeichnis\\
Deckblatt.tex & Deckblatt der Arbeit\\
Anleitung.tex & Dieser Anleitungstext (wird in der finalen Arbeit nicht ben\"otigt)\\
Beispielkapitel1.tex & Beispieltext (wird in der finalen Arbeit nicht ben\"otigt)\\
Beispielkapitel2.tex & Beispieltext (wird in der finalen Arbeit nicht ben\"otigt)\\
Anhang.tex & Anhang der Seminararbeit
\end{tabular}
\end{center}

\section{Das Deckblatt anpassen}
In der Datei \texttt{Deckblatt.tex} muss der Titel der Arbeit sowie der eigene Name eingetragen werden. Beim Titel k\"onnen (falls gew\"unscht) manuelle Zeilenumbr\"uche erzwungen werden, indem ein \textbackslash{}\textbackslash{} an das Ende der Zeile gesetzt wird:

\begin{lstlisting}[language=tex]
-Hier Thema eintragen-\\
-Hier Thema eintragen-\\
-Hier Thema eintragen-
\end{lstlisting}

\section{Grundstruktur der Arbeit anlegen}
Das Grundprinzip dabei ist, dass die verschiedenen Teile der Arbeit auf einzelne Dateien verteilt werden, welche dann jeweils \"ubersichtlich und leicht zu verwalten bleiben. In der Hauptdatei \texttt{Seminararbeit.tex} findet sich daf\"ur folgender Abschnitt:

\begin{lstlisting}[language=tex]
%%%%%%%%%%%%%%%%%%%%%%%%%%%%%%%%%%
% Hier eigene Dokumente einbinden
%%%%%%%%%%%%%%%%%%%%%%%%%%%%%%%%%%
\chapter{Anleitung}

Diese Vorlage besteht aus folgenden Dateien, welche alle im gleichen Verzeichnis liegen:

\begin{center}
\begin{tabular}{ll}
Dateiname & Inhalt\\\hline
Seminararbeit.tex & Struktur des Gesamtdokumentes\\
Quellen.bin & Literaturverzeichnis\\
Deckblatt.tex & Deckblatt der Arbeit\\
Anleitung.tex & Dieser Anleitungstext (wird in der finalen Arbeit nicht ben\"otigt)\\
Beispielkapitel1.tex & Beispieltext (wird in der finalen Arbeit nicht ben\"otigt)\\
Beispielkapitel2.tex & Beispieltext (wird in der finalen Arbeit nicht ben\"otigt)\\
Anhang.tex & Anhang der Seminararbeit
\end{tabular}
\end{center}

\section{Das Deckblatt anpassen}
In der Datei \texttt{Deckblatt.tex} muss der Titel der Arbeit sowie der eigene Name eingetragen werden. Beim Titel k\"onnen (falls gew\"unscht) manuelle Zeilenumbr\"uche erzwungen werden, indem ein \textbackslash{}\textbackslash{} an das Ende der Zeile gesetzt wird:

\begin{lstlisting}[language=tex]
-Hier Thema eintragen-\\
-Hier Thema eintragen-\\
-Hier Thema eintragen-
\end{lstlisting}

\section{Grundstruktur der Arbeit anlegen}
Das Grundprinzip dabei ist, dass die verschiedenen Teile der Arbeit auf einzelne Dateien verteilt werden, welche dann jeweils \"ubersichtlich und leicht zu verwalten bleiben. In der Hauptdatei \texttt{Seminararbeit.tex} findet sich daf\"ur folgender Abschnitt:

\begin{lstlisting}[language=tex]
%%%%%%%%%%%%%%%%%%%%%%%%%%%%%%%%%%
% Hier eigene Dokumente einbinden
%%%%%%%%%%%%%%%%%%%%%%%%%%%%%%%%%%
\chapter{Anleitung}

Diese Vorlage besteht aus folgenden Dateien, welche alle im gleichen Verzeichnis liegen:

\begin{center}
\begin{tabular}{ll}
Dateiname & Inhalt\\\hline
Seminararbeit.tex & Struktur des Gesamtdokumentes\\
Quellen.bin & Literaturverzeichnis\\
Deckblatt.tex & Deckblatt der Arbeit\\
Anleitung.tex & Dieser Anleitungstext (wird in der finalen Arbeit nicht ben\"otigt)\\
Beispielkapitel1.tex & Beispieltext (wird in der finalen Arbeit nicht ben\"otigt)\\
Beispielkapitel2.tex & Beispieltext (wird in der finalen Arbeit nicht ben\"otigt)\\
Anhang.tex & Anhang der Seminararbeit
\end{tabular}
\end{center}

\section{Das Deckblatt anpassen}
In der Datei \texttt{Deckblatt.tex} muss der Titel der Arbeit sowie der eigene Name eingetragen werden. Beim Titel k\"onnen (falls gew\"unscht) manuelle Zeilenumbr\"uche erzwungen werden, indem ein \textbackslash{}\textbackslash{} an das Ende der Zeile gesetzt wird:

\begin{lstlisting}[language=tex]
-Hier Thema eintragen-\\
-Hier Thema eintragen-\\
-Hier Thema eintragen-
\end{lstlisting}

\section{Grundstruktur der Arbeit anlegen}
Das Grundprinzip dabei ist, dass die verschiedenen Teile der Arbeit auf einzelne Dateien verteilt werden, welche dann jeweils \"ubersichtlich und leicht zu verwalten bleiben. In der Hauptdatei \texttt{Seminararbeit.tex} findet sich daf\"ur folgender Abschnitt:

\begin{lstlisting}[language=tex]
%%%%%%%%%%%%%%%%%%%%%%%%%%%%%%%%%%
% Hier eigene Dokumente einbinden
%%%%%%%%%%%%%%%%%%%%%%%%%%%%%%%%%%
\chapter{Anleitung}

Diese Vorlage besteht aus folgenden Dateien, welche alle im gleichen Verzeichnis liegen:

\begin{center}
\begin{tabular}{ll}
Dateiname & Inhalt\\\hline
Seminararbeit.tex & Struktur des Gesamtdokumentes\\
Quellen.bin & Literaturverzeichnis\\
Deckblatt.tex & Deckblatt der Arbeit\\
Anleitung.tex & Dieser Anleitungstext (wird in der finalen Arbeit nicht ben\"otigt)\\
Beispielkapitel1.tex & Beispieltext (wird in der finalen Arbeit nicht ben\"otigt)\\
Beispielkapitel2.tex & Beispieltext (wird in der finalen Arbeit nicht ben\"otigt)\\
Anhang.tex & Anhang der Seminararbeit
\end{tabular}
\end{center}

\section{Das Deckblatt anpassen}
In der Datei \texttt{Deckblatt.tex} muss der Titel der Arbeit sowie der eigene Name eingetragen werden. Beim Titel k\"onnen (falls gew\"unscht) manuelle Zeilenumbr\"uche erzwungen werden, indem ein \textbackslash{}\textbackslash{} an das Ende der Zeile gesetzt wird:

\begin{lstlisting}[language=tex]
-Hier Thema eintragen-\\
-Hier Thema eintragen-\\
-Hier Thema eintragen-
\end{lstlisting}

\section{Grundstruktur der Arbeit anlegen}
Das Grundprinzip dabei ist, dass die verschiedenen Teile der Arbeit auf einzelne Dateien verteilt werden, welche dann jeweils \"ubersichtlich und leicht zu verwalten bleiben. In der Hauptdatei \texttt{Seminararbeit.tex} findet sich daf\"ur folgender Abschnitt:

\begin{lstlisting}[language=tex]
%%%%%%%%%%%%%%%%%%%%%%%%%%%%%%%%%%
% Hier eigene Dokumente einbinden
%%%%%%%%%%%%%%%%%%%%%%%%%%%%%%%%%%
\input{Anleitung}
\input{Beispielkapitel1}
\input{Beispielkapitel2}
%%%%%%%%%%%%%%%%%%%%%%%%%%%%%%%%%%
%%%%%%%%%%%%%%%%%%%%%%%%%%%%%%%%%%
%%%%%%%%%%%%%%%%%%%%%%%%%%%%%%%%%%
\end{lstlisting}

F\"ur das erste eigene Kapitel der Seminararbeit kann dann z. B. eine leere Datei \texttt{Einleitung.tex} angelegt werden und mittels des \texttt{input} Befehls in das Hauptdokument eingebunden werden. Ebenso k\"onnen hier die Beispielkapitel und diese Anleitung aus dem Dokument entfernt werden, wenn diese nicht mehr ben\"otigt werden.

\section{Seminararbeit mit Texmaker erzeugen}
Um die Seminararbeit im Programm Texmaker bearbeiten und daraus ein PDF generieren zu k\"onnen, ist es am einfachsten, alle in der obigen Tabelle genannten Dateien in Texmaker zu \"offnen. Anschlie\ss{}end in Texmaker die Datei \texttt{Seminararbeit.tex} aufrufen und diese \"uber das Men\"u als Hauptdokument festlegen:
\begin{center}
\includegraphics[width=0.5\textwidth]{Bilder/anleitung2.png}
\end{center}

\input{Beispielkapitel1}
\input{Beispielkapitel2}
%%%%%%%%%%%%%%%%%%%%%%%%%%%%%%%%%%
%%%%%%%%%%%%%%%%%%%%%%%%%%%%%%%%%%
%%%%%%%%%%%%%%%%%%%%%%%%%%%%%%%%%%
\end{lstlisting}

F\"ur das erste eigene Kapitel der Seminararbeit kann dann z. B. eine leere Datei \texttt{Einleitung.tex} angelegt werden und mittels des \texttt{input} Befehls in das Hauptdokument eingebunden werden. Ebenso k\"onnen hier die Beispielkapitel und diese Anleitung aus dem Dokument entfernt werden, wenn diese nicht mehr ben\"otigt werden.

\section{Seminararbeit mit Texmaker erzeugen}
Um die Seminararbeit im Programm Texmaker bearbeiten und daraus ein PDF generieren zu k\"onnen, ist es am einfachsten, alle in der obigen Tabelle genannten Dateien in Texmaker zu \"offnen. Anschlie\ss{}end in Texmaker die Datei \texttt{Seminararbeit.tex} aufrufen und diese \"uber das Men\"u als Hauptdokument festlegen:
\begin{center}
\includegraphics[width=0.5\textwidth]{Bilder/anleitung2.png}
\end{center}

\input{Beispielkapitel1}
\input{Beispielkapitel2}
%%%%%%%%%%%%%%%%%%%%%%%%%%%%%%%%%%
%%%%%%%%%%%%%%%%%%%%%%%%%%%%%%%%%%
%%%%%%%%%%%%%%%%%%%%%%%%%%%%%%%%%%
\end{lstlisting}

F\"ur das erste eigene Kapitel der Seminararbeit kann dann z. B. eine leere Datei \texttt{Einleitung.tex} angelegt werden und mittels des \texttt{input} Befehls in das Hauptdokument eingebunden werden. Ebenso k\"onnen hier die Beispielkapitel und diese Anleitung aus dem Dokument entfernt werden, wenn diese nicht mehr ben\"otigt werden.

\section{Seminararbeit mit Texmaker erzeugen}
Um die Seminararbeit im Programm Texmaker bearbeiten und daraus ein PDF generieren zu k\"onnen, ist es am einfachsten, alle in der obigen Tabelle genannten Dateien in Texmaker zu \"offnen. Anschlie\ss{}end in Texmaker die Datei \texttt{Seminararbeit.tex} aufrufen und diese \"uber das Men\"u als Hauptdokument festlegen:
\begin{center}
\includegraphics[width=0.5\textwidth]{Bilder/anleitung2.png}
\end{center}

\input{Beispielkapitel1}
\input{Beispielkapitel2}
%%%%%%%%%%%%%%%%%%%%%%%%%%%%%%%%%%
%%%%%%%%%%%%%%%%%%%%%%%%%%%%%%%%%%
%%%%%%%%%%%%%%%%%%%%%%%%%%%%%%%%%%
\end{lstlisting}

F\"ur das erste eigene Kapitel der Seminararbeit kann dann z. B. eine leere Datei \texttt{Einleitung.tex} angelegt werden und mittels des \texttt{input} Befehls in das Hauptdokument eingebunden werden. Ebenso k\"onnen hier die Beispielkapitel und diese Anleitung aus dem Dokument entfernt werden, wenn diese nicht mehr ben\"otigt werden.

\section{Seminararbeit mit Texmaker erzeugen}
Um die Seminararbeit im Programm Texmaker bearbeiten und daraus ein PDF generieren zu k\"onnen, ist es am einfachsten, alle in der obigen Tabelle genannten Dateien in Texmaker zu \"offnen. Anschlie\ss{}end in Texmaker die Datei \texttt{Seminararbeit.tex} aufrufen und diese \"uber das Men\"u als Hauptdokument festlegen:
\begin{center}
\includegraphics[width=0.5\textwidth]{Bilder/anleitung2.png}
\end{center}
