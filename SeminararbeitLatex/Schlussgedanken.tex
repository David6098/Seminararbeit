\chapter{Gesamtfazit und Schlussgedanken}
Wenn man die Zeit, welche ich dafür verwendet habe ein funktionierendes Neuronales Netzwerk zu entwickeln und programmieren, mit der Zeit gegenüberstellt, die für ich den Algorithmus in \ref{algor} gebraucht habe, so stellt man fest, das die eine Aufgabe über 20 Stunden und die andere nur um die 5 Minuten gedauert hat, wobei das Ergebnis der einfacheren Aufgabe besser ist, als das der Komplizierteren. Folglich kann man daraus schließen das man für so eine einfache Aufgabe, wie das Computerspiel Space Invaders zu schlagen, keine Methoden anwenden sollte, die komplizierter sind, als solche welche in fünf Minuten zum Ziel führen. Man muss auch bedenken, dass diese einfacheren Methoden einfacher zu verbessern, warten und auch verständlicher sind. Im Nachhinein hätte das Ziel für meine Seminararbeit nicht lauten sollen: \glqq Versuch etwas Neues, dass noch niemand getan hat\grqq{}, sondern  \glqq Benutze das von dem du weißt es funktioniert\grqq{}.\\
Zudem bin ich einige Tage vor der Abgabe auf OpenAI Gym gestoßen \cite[vgl.][]{1606.01540}. Ein Projekt auf den man seine KI unkompliziert implementieren und verschiedene Lernalgorithmen miteinander vergleichen kann. Leider funktioniert die Bibliothek, welche dafür verwendet wird, nur auf LINUX und MacOs, aber nicht Windows, ansonsten hätte ich eine einfachere Möglichkeit mein Projekt mit  Anderen zu vergleichen.